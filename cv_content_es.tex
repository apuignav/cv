% cv_content.tex
% Content of the CV

% Aesthetics
\usepackage{xspace}
\def\epfl{\'{E}cole Polytechnique F\'{e}d\'{e}rale de Lausanne\xspace}
\def\myast{\text{\,*}}

% Start
\begin{document}
\header{albert}{puig}
       {físico de altas energías}

\begin{aside}
  \section{sobre mi}
    Albert Puig Navarro\\
    Nacido en 26-07-1984\\
    ~\\
    Avenue Chailly 22\\
    1012 Lausanne\\
    Suiza\\
    ~\\
    \href{mailto:albert.puig@epfl.ch}{albert.puig@epfl.ch}\\
    \href{https://github.com/apuignav}{github://apuignav}\\
    %+41 (0)216939808
  %\section{languages}
    %bilingual french/english
    %spanish \& italian notions
\ifthenelse{\boolean{includeprogramming}}
{  \section{programación}
    Python, C++,\\
    Fortran, PVSS,\\
    MySQL, VB.NET\\
}{}
    \section{intereses}
    física de altas energías\\
    física de $B$\\
    técnicas de trigger\\
    análisis de datos\\
    computación\\

\end{aside}

\section{educación}

  \cventry
    {desde 2012}
    {Investigador postdoctoral}
    {\epfl(Suiza)}
    {}
  \cventry
    {2007--2012}
    {Doctorado en Física}
    {Universitat de Barcelona (España)}
    {\emph{First measurements of radiative $B$ decays in LHCb}, supervisado por Ricardo Graciani}
  \cventry
    {2007}
    {Master en Astrofísica, Física de Partículas y Cosmología}
    {Universitat de Barcelona (España)}
    {\emph{SPD monitoring}, supervisado por  Ricardo Graciani}
  \cventry
    {2002--2007}
    {Licenciatura en Física}
    {Universitat de Barcelona (España)}
    {}

\section{actividades de investigación}
Soy miembro de la colaboración LHCb desde 2007 y soy coautor de sus 193 publicaciones.

%Since 2012 I am the trigger liaison for the LHCb Rare Decays working group and since 2013 I am the convener of the LHCb Radiative Decays working group.

\subsection{desintegraciones radiativas}
    \cventry
    {desde 2013}
    {Coordinador del grupo de Radiative Decays en LHCb}
    {}
    {}
    \cventry
    {desde 2012}
    {Estudio de la polarización del fotón en $\mathbf{B^+\!\to K^+\pi^-\pi^+\gamma}$}
    {}
    {LHCb collaboration, Aaij, R. \textit{et al.}, \textit{Observation of photon polarization in the $b \to s\gamma$ transition}, Phys. Rev. Lett. \textbf{112} (2014) $161801$}
    \cventry
    {2011--2012}
    {Estudi de la asimetria de $\mathbf{CP}$ en $\mathbf{B^0\!\to K^{\myast0}\gamma}$}
    {}
    {LHCb collaboration, Aaij, R. \textit{et al.}, \textit{Measurement of the ratio of branching fractions ${\cal B}(B^0\!\to K^{\myast0} \gamma) /$ ${\cal B}(B^0_s\!\to \phi \gamma)$ and the direct $CP$ asymmetry in $B^0\!\to K^{\myast0} \gamma$}, Nucl. Phys. \textbf{B867} (2013) $1$--$18$}
    \cventry
    {2011--2012}
    {Estudio de la razón de fracciones de desintegración de $\mathbf{B^0\!\to K^{\myast0}\gamma}$ y $\mathbf{B^0_s\!\to \phi\gamma}$}
    {}
    {LHCb collaboration, Aaij, R. \textit{et al.}, \textit{Measurement of the ratio of branching fractions $\mathcal{B}(B^0\!\to K^{\myast0}\gamma)/$ $\mathcal{B}(B_s^0\to\phi\gamma)$}, Phys. Rev. \textbf{D85} (2012) $112013$}

\subsection{trigger}
    \cventry
    {desde 2012}
    {Enlace del grupo de trigger en el grupo de trabajo de Rare Decays}
    {}
    {}
    \cventry
    {desde 2011}
    {Diseño e implementación de estrategias de trigger para desintegraciones radiativas de mesones $\mathbf{B}$}
    {}
    {Aaij, R. \textit{et al.}, \textit{The LHCb trigger and its performance in 2011}, JINST \textbf{8} (2013) P$04022$\\
     Puig, A., \textit{The HLT$2$ Radiative Topological Lines}, LHCb-PUB-$2012$-$002$. CERN-LHCb-PUB-$2012$-$002$ (2012)}

\subsection{computación}
    \cventry
    {2012--2014}
    {Responsable y desarrollador principal del \emph{software} de analisis de \emph{test beams} para el detector SciFi}
    {}
    {}
    \cventry
    {2011--2012}
    {Integración del \emph{middleware} DIRAC con el portal WS-PGRADE para la ejecución de tareas gUSE en un entorno de computación distribuida}
    {}
    {Puig, A. \emph{et al.}, \textit{Integration of the gUSE/WS-PGRADE and InSilicoLab portals with DIRAC}, J. Phys. Conf. Ser. \textbf{396} (2012) $032088$}
    \cventry
    {2010}
    {Análisis de patrones de uso del GRID de los usuarios de LHCb a través del estudio de la base de datos Accounting de DIRAC}
    {}
    {Casaj{\'u}s, A. \emph{et al}, \textit{The LHCb experience on the Grid from the DIRAC accounting data}, J. Phys. Conf. Ser. \textbf{331} (2011) $072059$}
    \cventry
    {2007--2010}
    {Diseño e implementación de un sistema para la reconstrucción de sucesos en la LHCb Online farm}
    {}
    {Puig, A. and Frank, M., \textit{Event reconstruction in the LHCb online cluster}, J. Phys. Conf. Ser. \textbf{219} (2010) $022020$}

\subsection{i\&d de hardware}
    \cventry
    {2012--2014}
    {Estudios de I\&D de SiPM para el Upgrade de LHCb, incluyendo algoritmos de \emph{clustering} y pruebas con rayos cósmicos}
    {}
    {Bay, A. \emph{et al.}, \textit{Viability Assessment of a Scintillating Fibre Tracker for the LHCb Upgrade}, LHCb-PUB-2014-015. CERN-LHCb-PUB-2014-015. LHCb-INT-2013-004 (2014)}
    \cventry
    {2008--2011}
    {Calibración fina del Calorímetro Electromagnético de LHCb con $\mathbf{\pi^0}$}
    {}
    {Belyaev, I. \emph{et al.}, \textit{Kali: The framework for fine calibration of the LHCb electromagnetic calorimeter}, J. Phys. Conf. Ser. \textbf{331} (2011) $032050$}
    \cventry
    {2007--2008}
    {Diseño e implementación del sistema de monitorización del subdetector SPD en LHCb}
    {}
    {Abellan Beteta, C. \emph{et al.}, \textit{Time alignment of the front end electronics of the LHCb calorimeters.}, J. Instrum. \textbf{7} (2011) P$08020$}

\ifthenelse{\boolean{includeconferences}}
{
\section{conferencias y seminarios}

    \cventry
        {2014}
        {Radiative electroweak penguins at LHCb}
        {Edimburgo (Reino Unido)}
        {Charla en \emph{15th International Conference on B-Physics at Frontier Machines}, 14--18 de julio}
    \cventry
        {2014}
        {Observation of photon polarization in the $\mathbf{b\to s\gamma}$ transition at LHCb}
        {CERN (Suiza)}
        {\emph{LHC Seminar}, 18 de marzo}
    \cventry
        {2013}
        {The LHCb Trigger System: Performance and Outlook}
        {Seoul (Korea del Sur)}
        {Charla en \emph{IEEE Nuclear Science Symposium} 2013, 27 de octubre a 2 de noviembre}
    \cventry
        {2013}
        {Radiative $\mathbf{B}$ decays at LHCb}
        {Estocolmo (Suecia)}
        {Charla en \emph{EPSHEP} $2013$, 18--24 de julio}
    \cventry
        {2013}
        {Rare beauty and charm decays}
        {Ann Arbor, Michigan (USA)}
        {Charla en \emph{KAON}$2013$, 29 de abril a 1 de mayo}
    \cventry
        {2012}
        {Radiative $\mathbf{B}$ decays in LHCb}
        {Benasque (España)}
        {Charla en \emph{International Meeting on Fundamental Physics}, 24 de mayo a 3 de junio}
    \cventry
        {2012}
        {Integration of the gUSE/WS-PGRADE and InSilicoLab portals with DIRAC}
        {Nueva York (USA)}
        {Poster en \emph{Computer in High Energy and Nuclear Physics (CHEP) 2012}, 21--25 de mayo}
    \cventry
        {2012}
        {Radiative $\mathbf{B}$ decays in LHCb}
        {Marseille (Francia)}
        {Seminario en el CPPM, 22 de marzo}
    \cventry
        {2012}
        {Results on radiative $\mathbf{B}$ decays}
        {Lake Louise (Canada)}
        {Charla en \emph{Lake Louise Winter Institute}, 19--25 de febrero}
    \cventry
        {2011}
        {Radiative $\mathbf{B}$ decays at LHCb}
        {Barcelona (España)}
        {Charla en \emph{III CPAN Days}, 2--4 de noviembre}
    \cventry
        {2011}
        {Exclusive rare $\mathbf{B}$ decays at LHCb}
        {Durham (Reino Unido)}
        {Charla en \emph{IPPP Flavour and the 4th generation}, 14--16 de septiembre}
    \cventry
        {2010}
        {The LHCb experience on the Grid from the DIRAC Accounting data}
        {Taipei (Taiwan)}
        {Poster en \emph{Computing in High Energy and Nuclear Physics (CHEP) 2010}, 18--22 de octubre}
    \cventry
        {2010}
        {Kali: the framework for fine calibration of LHCb Calorimeter}
        {Taipei (Taiwan)}
        {Poster en \emph{Computing in High Energy and Nuclear Physics (CHEP) 2010}, 18--22 de octubre}
    \cventry
        {2009}
        {Calibraci\'{o}n fina del calor\'{i}metro electromagn\'{e}tico de LHCb}
        {Ciudad Real (España)}
        {Charla en \emph{XXXII Reuni\'{o}n Bienal de la RSEF}, 7--11 de septiembre}
    \cventry
        {2009}
        {Event reconstruction in the LHCb Online cluster}
        {Praga (República Checa)}
        {Charla en \emph{Computing in High Energy and Nuclear Physics (CHEP) 2009}, 23--27 de marzo}
}{}

\ifthenelse{\boolean{includeteaching}}
{
\section{docencia}
    \cventry
        {2014}
        {Supervisor de proyecto TP4}
        {\epfl(Suiza)}
        {Violaine Bellée, \emph{Search for the $B^0\rightarrow \phi K^{\myast0}\gamma$ and $B_s^0\rightarrow \phi \phi\gamma$ decays at LHCb}, semestre de primavera}
    \cventry
        {2013--2014}
        {Supervisor de tesina de Master}
        {\epfl(Suiza)}
        {Eleonie Van Schreven, \emph{Photon polarisation sensitivity in $B^0_s\rightarrow \phi\gamma$ decays at LHCb}, semestre de otoño}
    \cventry
        {2013}
        {Supervisor de proyecto TP4}
        {\epfl(Suiza)}
        {Brice Maurin, \emph{Improving the selection for the decay $B^0\rightarrow K^{\myast0} \gamma$}, semestre de primavera}
    \cventry
        {2012--2013}
        {Supervisor de tesina de Master}
        {\epfl(Suiza)}
        {Isaure Leboucq, \emph{Observation of the decay $B^+\rightarrow K^+\pi^-\pi^-\gamma$ at LHCb}, semestre de otoño}
    \cventry
        {2011-2012}
        {Laboratorio de Física}
        {Universitat de Barcelona (España)}
        {Facultad de Química, semestre de otoño}
    \cventry
        {2011}
        {Taller de Altas Energ\'{i}as}
        {Universidad del País Vasco, Bilbao (España)}
        {Encargado de ejercicios del curso \emph{ATLAS Physics}, con Peter Jenni (CERN).\\
         Encargado de ejercicios del curso \emph{Standard Model}, con Nuria Rius (IFIC).}
        %{July 11--22}
    \cventry
        {2010--2011}
        {Fundamentos de Laboratorio}
        {Universitat de Barcelona (España)}
        {Facultad de Física, semestre de primavera}
    \cventry
        {2009--2010}
        {Laboratorio de Física Moderna}
        {Universitat de Barcelona (España)}
        {Facultad de Física, semestre de otoño}
}{}
    
\section{actualizado}
30 de septiembre de 2014

\includegraphics{signature.jpg}
%\section{references}
%Available upon request

%\section{computer skills}

%\section{selected publications}

%\section{LHCb papers}

\end{document}

% EOF
